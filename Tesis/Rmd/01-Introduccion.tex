% Options for packages loaded elsewhere
\PassOptionsToPackage{unicode}{hyperref}
\PassOptionsToPackage{hyphens}{url}
%
\documentclass[
]{krantz}
\usepackage{lmodern}
\usepackage{amsmath}
\usepackage{ifxetex,ifluatex}
\ifnum 0\ifxetex 1\fi\ifluatex 1\fi=0 % if pdftex
  \usepackage[T1]{fontenc}
  \usepackage[utf8]{inputenc}
  \usepackage{textcomp} % provide euro and other symbols
  \usepackage{amssymb}
\else % if luatex or xetex
  \usepackage{unicode-math}
  \defaultfontfeatures{Scale=MatchLowercase}
  \defaultfontfeatures[\rmfamily]{Ligatures=TeX,Scale=1}
\fi
% Use upquote if available, for straight quotes in verbatim environments
\IfFileExists{upquote.sty}{\usepackage{upquote}}{}
\IfFileExists{microtype.sty}{% use microtype if available
  \usepackage[]{microtype}
  \UseMicrotypeSet[protrusion]{basicmath} % disable protrusion for tt fonts
}{}
\makeatletter
\@ifundefined{KOMAClassName}{% if non-KOMA class
  \IfFileExists{parskip.sty}{%
    \usepackage{parskip}
  }{% else
    \setlength{\parindent}{0pt}
    \setlength{\parskip}{6pt plus 2pt minus 1pt}}
}{% if KOMA class
  \KOMAoptions{parskip=half}}
\makeatother
\usepackage{xcolor}
\IfFileExists{xurl.sty}{\usepackage{xurl}}{} % add URL line breaks if available
\IfFileExists{bookmark.sty}{\usepackage{bookmark}}{\usepackage{hyperref}}
\hypersetup{
  pdftitle={Modelos de demanda probabilística en sistemas dinámicos complejos},
  pdfauthor={Alejandro Verri Kozlowski},
  hidelinks,
  pdfcreator={LaTeX via pandoc}}
\urlstyle{same} % disable monospaced font for URLs
\usepackage{longtable,booktabs}
\usepackage{calc} % for calculating minipage widths
% Correct order of tables after \paragraph or \subparagraph
\usepackage{etoolbox}
\makeatletter
\patchcmd\longtable{\par}{\if@noskipsec\mbox{}\fi\par}{}{}
\makeatother
% Allow footnotes in longtable head/foot
\IfFileExists{footnotehyper.sty}{\usepackage{footnotehyper}}{\usepackage{footnote}}
\makesavenoteenv{longtable}
\usepackage{graphicx}
\makeatletter
\def\maxwidth{\ifdim\Gin@nat@width>\linewidth\linewidth\else\Gin@nat@width\fi}
\def\maxheight{\ifdim\Gin@nat@height>\textheight\textheight\else\Gin@nat@height\fi}
\makeatother
% Scale images if necessary, so that they will not overflow the page
% margins by default, and it is still possible to overwrite the defaults
% using explicit options in \includegraphics[width, height, ...]{}
\setkeys{Gin}{width=\maxwidth,height=\maxheight,keepaspectratio}
% Set default figure placement to htbp
\makeatletter
\def\fps@figure{htbp}
\makeatother
\setlength{\emergencystretch}{3em} % prevent overfull lines
\providecommand{\tightlist}{%
  \setlength{\itemsep}{0pt}\setlength{\parskip}{0pt}}
\setcounter{secnumdepth}{5}
\ifluatex
  \usepackage{selnolig}  % disable illegal ligatures
\fi

\title{Modelos de demanda probabilística en sistemas dinámicos complejos}
\usepackage{etoolbox}
\makeatletter
\providecommand{\subtitle}[1]{% add subtitle to \maketitle
  \apptocmd{\@title}{\par {\large #1 \par}}{}{}
}
\makeatother
\subtitle{Parte A: Preliminares}
\author{Alejandro Verri Kozlowski}
\date{2021-04-30}

\begin{document}
\maketitle

{
\setcounter{tocdepth}{1}
\tableofcontents
}
\hypertarget{introducciuxf3n}{%
\section*{Introducción}\label{introducciuxf3n}}
\addcontentsline{toc}{section}{Introducción}

\hypertarget{estado-actual-de-la-pruxe1ctica}{%
\subsection*{Estado Actual de la Práctica}\label{estado-actual-de-la-pruxe1ctica}}
\addcontentsline{toc}{subsection}{Estado Actual de la Práctica}

La ingeniería sísmica es el vínculo entre las ciencias de la tierra y la ingeniería civil. El principal aporte de la ingeniería sísmica al diseño en ingeniería es la definición de las acciones sísmicas (cargas, desplazamientos)en términos de su intensidad y frecuencia de ocurrencia durante la vida útil de una estructura.

La respuesta sísmica de un sistema es el conjunto de parámetros que caracteriza la respuesta dinámica de un sistema dinámico cuando es sometido sujeto a movimientos sísmicos en la superficie de fundación y puede interpretarse como una expresión del movimiento sísmico en el sistema, en términos de fuerzas, distorsiones, esfuerzos, etc.

El sitio en donde se fundan las estructuras e instalaciones, puede considerarse en sí mismo como un sistema dinámico que responde cuando es sometido a movimientos sísmicos a nivel de roca basal. Este sistema tendrá propiedades dinámicas que dependeRÁNn de las propiedades mecánicas de los estratos de suelo apoyado sobre la roca basal.

Los movimientos sísmicos a nivel de la roca basal constituyen a su vez la respuesta dinámica de un sistema mucho más complejo de propagación de ondas elásticas sísmicas a través de las dierentes capas de la corteza terrestre, originado por el movimiento súbito de placas tectónicas que constituyen los eventos sísmicos.

La demanda es un parámetro de la respuesta sísmica del sistema (por ejemplo, el esfuerzo cortante de una columna, o el asentamiento de un terraplén inducidos por sismos) que puede asociarse a un objetivo de diseño y puede entenderse como una expresión del movimiento sísmico en ese sistema dinámico particular.

En un escenario determinístico, un modelo de estimación de la demanda sísmica requiere como mínimo cuatro pasos:

\begin{itemize}
\tightlist
\item
  La estimación de la intensidad del movimiento sísmico a nivel de roca basal en el sitio del proyecto.
\item
  La estimación de la intensidad del movimiento sísmico a nivel de superficie, para un sitio con características geotécnicas dadas.
\item
  La estimación de la respuesta sísmica que produce ese movimiento sísmico regional en el sitio de emplazamiento.
\item
  La identificación de los parámetros de la respuesta sísmica (demanda) para la cual quiero asegurar el cumplimiento de algún criterio de diseño.
\end{itemize}

Las filosofías de diseño sísmico basadas en la capacidad, que se emplean en la mayoría de los códigos y reglamentos de diseño estructural, limitan la demanda en términos de estados límites últimos o estados límites de servicio y deben segurar que \(D \leq C\).

Las filosofías de diseño basadas en la performance (PBEE) por otra parte, limitan la demanda en términos de un conjunto de objetivos específicos de desempeño (distorsiones máximas, asentamientos, etc) que deben cumplirse para diferentes niveles de servicio (construcción, operación, cierre) \(\left\{ D_{k}\leq d^{\ast }_{k}\right\}\).

En cualquiera de los marcos de análisis, la ocurrencia de eventos sísmicos es un fenñómeno de naturaleza aleatoria y la demanda y la intensidad sísmica serán por lo tanto variables aleatorias y los estados límite de capacidad o los objetivos de performance sólo podrán cumplirse de manera probabilística. Luego, el diseño óptimo de un sistema deberá asegurar que la probabilidad de excedencia de la demanda sísmica dsea menor a cierto valor objetivo \(p^*\). Para lograr este objetivo de diseño basado en la confiabilidad, el modelo deberá incluir dos etapas adicionales

\begin{itemize}
\tightlist
\item
  La estimación de la tasa anual de ocurrencia de eventos mayores a una cierta intensidad en una región dada.
\item
  La estimación de la tasa anual en que la demanda sísmica, excede un cierto estado límite
\end{itemize}

Para poder identificar las limitaciones y desafíos del estado actual de la práctica, se plantea el problema a traves de los pasos que se requieren para el desarrollo de un modelo de estimación de demanda sísmica \(G_D(d^*)\) de un sistema dinámico general.

\hypertarget{planteo-del-problema}{%
\subsubsection*{Planteo del Problema}\label{planteo-del-problema}}
\addcontentsline{toc}{subsubsection}{Planteo del Problema}

Si asumimos que la demanda sísmica \(D\) está relacionada exclusivamente con la ocurrencia de eventos sísmicos durante un período de exposición \(T_e\), y que la ocurrencia de estos eventos sigue un proceso de Poisson, la demanda sísmica dependerá del período de exposición \(T_e\) y la confiabilidad del diseño puede expresarse numéricamente según \[P_{T_e}\left [D > d^*  \right ]\approx 1-e^{-\lambda_D(d^*) T_e}\leq p^*\] El parámetro de Poisson \(\lambda_D\) es la tasa anual en que la demanda \(D\) excede un cierto umbral \(d^*\) y depende de la probabilidad de excedencia \(P\left [D>d^* \right ]\) de un evento sísmico, y del número de eventos por año \(\nu_o\) capaces de producir daño.

\[ \begin{equation}
\lambda_D(d^*) \approx \nu_o P\left [D>d^* \right ] 
\end{equation}\]

Entendiendo que la respuesta sísmica de un sistema sistema será dependiente del movimiento sísmico del sitio del proyecto, la probabilidad de excedencia \(P\left [ D>d^* \right ]\) debe poder condicionarse a algunas características del movimiento sísmico. El planteo tradicional es caracterizar el movimiento sísmico a partir de una medida escalar de intensidad escalar \(I=i^*\). Bajo esta hipótesis, la probabilidad de excedencia puede estimarse mediante el teorema de la probabilidad total

\[ \begin{equation} 
P\left [ D>d^* \right ] \approx \sum^{}_{all\  i^{\ast }}  P\left [D > d^*|I \right ] \ P\left [I=i^*  \right ] \approx G_D(d^*)
\end{equation}\] La frecuencia con la que ocurren eventos de intensidad \(I=i^*\) puede obtenerse a partir de la probabilidad de excedencia \(P\left [ I>i^* \right ]\) para todos los escenarios posibles de eventos sísmicos en el sitio. La intensidad sísmica del sitio \(i^*\) dependerá en general de la distancia del sitio al punto más cercano de la falla \(r^*\) y de la magnitud \(m^*\) del evento sísmico, el cual está directamente relacionado con el tamaño del área de falla.

Debido a la naturaleza aleatoria asociada a la ubicación de los hipocentros de un terremoto y al tamaño del área de ruptura asociada a durante un evento sísmico, la distancia a la falla \(R\) y la magnitud \(M\) son variables aleatorias que expresan de manera simplificada la variabilidad temporal y espacial, respectivamente, del movimiento sísmico y la intensidad \(i^*\) asociada a un sitio queda definida en términos de una función de excedencia mediante el Teorema de la Probabilidad Total (TPT)

\[ \begin{equation} 
P\left [ I>i^* \right ] \approx \sum^{}_{all\  m^{\ast},r^{\ast}}  P\left [I > i^*|\ M,R \right ] \ P\left [M=m^*,R=r^*  \right ] \approx G_I(i^*)
\end{equation}\]

En general, las variables aleatorias \(R\) y \(M\) suelen considerarse \emph{iid} y \(P\left [M=m^*,R=r^* \right ] \approx P\left [M=m^*\right ] P\left [R=r^*\right ]\) . Sin embargo, existen algunos escenarios en donde puedan estar fuertemente correlacionadas. Por ejemplo, en sitios cercanos a fallas activas, la distancia al punto más cercano de la falla \(R\) suele estar correlacionada con la ubicación del área de ruptura y la función de distribución de las distancias requiere del conocimiento de todas las posibles áreas de ruptura, por lo que \(P\left [R=r^*,R=r^* \right ] \approx P\left [R=r^*|M=m^*\right ] P\left [M=m^*\right ]\)

Luego, la probabilidad de excedencia de un evento sísmico de intensidad \(I=i^*\) queda determinada según la función de densidad acumulada \(G_I(i^*)\) según

\[ \begin{equation}  G_I(i^*) \approx \int^{r_{max}^*}_{r_{min}^*}  \int^{m_{max}^*}_{m_{min}^*} G_{I|R,M}(i^*,r,m) \ f_{R|M}(r,m) \ f_M(m) \ dm \ dr 
\end{equation}\]

La función de densidad \(f_I(i^*)\) que define la probabilidad ocurrencia de un evento sísmico de intensidad \(I=i^*\) queda determinada según \[ f_I(i^*) \approx \left|\frac{\partial G_I(i^*)}{\partial i}\right | \approx \int^{r_{max}^*}_{r_{min}^*}  \int^{m_{max}^*}_{m_{min}^*} \left|\frac{\partial G_{I|R,M}(i^*,r,m)}{\partial i} \right| f_{R|M}(r,m)  f_M(m)  dm dr  \]

Finalmente, la probabilidad de que la demanda sísmica exceda un valor límite \(d^*\) dado que ocurrió un evento de intensidad \(I=i\) se puede obtener mediante la expresión general:

\[ \begin{equation} 
G_D(d^*) \approx \int^{i_{max}^*}_{i_{min}^*} \int^{r_{max}^*}_{r_{min}^*}  \int^{m_{max}^*}_{m_{min}^*} G_{D|I}(d^*,i) \left | \frac{\partial G_{I|R,M}(i,r,m)}{\partial i} \right |f_{R|M}(r,m) \ f_M(m)  \ dm \ dr \ di
\end{equation}\]

El paso siguiente es efectuar algunas hipótesis sobre la ley de distribución de las variables aleatorias condicionadas \(D|I\) y \(I|R,M\). Los modelos donde las formas funcionales de las distribuciones de probabilidad se conocen, se denominan modelos paramétricos y sólo requieren la determinación de un número finito de parámetros que parametrizan o ajustan esas formas funcionales. Si se asume conocida la forma funcional de la ley de distribución para las demanda y la intensidad, las funciones de distribución de demanda \(f_{D|I}(i)\) e intensidad \(f_{I|R,M}(r,m)\) puede obtenerse a partir de estimadores de la mediana condicional y la varianza.

El paso siguiente es buscar una transformación de las variables aleatorias tal que la variable transformada tenga una ley de distribución aproximadamente Normal. La transformación más simple que se ajusta a los problemas de intensidad y demanda asociadas a la ocurrencia de eventos sísmicos, es la transformación logarítmica, que en la práctica equivale a asumir una ley de distribución log-normal para las variables aleatorias. Luego centrar y normalizar, las nuevas variables aleatorias de la demanda e intensidad quedan determinadas según

\[ \begin{equation} 
\epsilon_{D}(d^*,i) \approx  \frac{ln\ d^*-ln [\eta_{D|I}(i)]}{\sigma_{ln \ D} }  \ \ \ \ \ \ \ \ \ \ \ \ \ \ \ \ \ \epsilon_{I}(i^*,m,r) \approx  \frac{ln \ i^*-ln[\eta_{I|M,R}(m,r)]}{\sigma_{ln \ I} } \leq \epsilon_{T}
\end{equation}\]

El término \(ln [\eta_{I|M,R}(i)] \approx \mu_{ln I|M,R}(i)\) es la mediana condicional de un modelo de predicción del movimiento sísmico (GMPE) y ajusta las intensidades obtenidas de una selección de registros de aceleraciones para diferentes escenarios sísmicos de magnitud y distancia \({M,R}\). Por otra parte, el término \(\eta_{D|I}(i)=exp(\mu_{ln \ D|I}(i))\) es la mediana condicional de la demanda sísmica, obtenida numéricamente mediante un modelo numérico o un proxy analítico.

En teoría, la mediana condicional \(\mu_{ln D|I}(i)\) y la varianza \(\sigma_{ln \ D|I}\) siempre podrán estimarse mediante la media muestral de las demandas obtenidas en un conjunto de registros sísmicos . Si se analizan \(n\) registros sísmicos escalados a una intensidad objetivo \(I=i^*\), y se conocen \(n_r\) valores de la demanda sísmica condicional \(D|I = \{d_1^*,d_2^*,...,d_{n}^* \}\), en el caso mas simple de un modelo de regresión lineal la mediana y la varianza quedan determinadas según

\[ \begin{equation} 
\eta_{D|I}(i^*) \approx \prod^{n_r}_{k=1}{exp [d_k^*/n]} \ \ \ \ \ \ \ \ \ \ \ \ \ \ \ \ \ \sigma_{lnD|I}^2 \approx \frac {1}{n} \ \sum^{n}_{1}{(ln \ d_k^*)^2} - ln [\eta_{D|I}(i^*)]^2
\end{equation}\]

Una de las ventajas de esta transformación es que las CDF \(G_{I|R,M}(i^*,r,m)\) y \(G_{D|I}(d^*,i)\) pueden definirse explícitamente en términos de la función de densidad normal \(\phi(\epsilon^*)\) y la función de densidad acumulada complementaria normal \(\tilde\Phi(\epsilon^*)=1-\Phi(\epsilon^*)\) donde \(\epsilon^*\) es una variable aleatoria normal normalizada, con valor medio \(\mu_{\epsilon}=0\) y desvío \(\sigma_{\epsilon}=1\)

Cuando se conocen las funciones de distribución de la intensidad y la demanda, la ecuación de diseño del sistema dinámico queda determinada mediante la resolución numérica de la siguiente integral:

\[ \begin{equation}  
G_D(d^*) \approx \int^{i_{max}^*}_{i_{min}^*} \int^{r_{max}^*}_{r_{min}^*}  \int^{m_{max}^*}_{m_{min}^*}  \ \frac{\tilde\Phi \left[ \epsilon_{D}(d^*,i) \right] \phi \left[\epsilon_{I}(i,m,r) \right]  \ f_{R|M}(r,m) \ f_M(m)}{ i \ \sigma_{ln I}} \  dm \ dr \ di \leq \frac {-ln \left(1-p^* \right)}{\nu_o \ T_E}  
\end{equation}\]

Las variables aleatorias normalizadas \(\epsilon_{D}(d^*,i)\) y \(\epsilon_{I}(i^*,m,r)\) dependen indirectamente de los términos de error \(\sigma_{ln \ D}\) y \(\sigma_{ln \ I}\) controlan toda la incertidumbre del modelo de estimacion de demanda \(G_D(d^*)\)

Las ecuaciones anteriores, presentan el estado actual de la práctica en el diseño sísmico basado en performance de estructuras elásticas con respuestas asemejables a un SDOF. En el capítulo que sigue, se resumen algunas de las limitaciones de este modelo de análisis para sistemas dinámicos más generales

\hypertarget{limitaciones-del-estado-actual-de-la-pruxe1ctica}{%
\subsection*{Limitaciones del estado actual de la práctica}\label{limitaciones-del-estado-actual-de-la-pruxe1ctica}}
\addcontentsline{toc}{subsection}{Limitaciones del estado actual de la práctica}

\hypertarget{carcterizaciuxf3n-del-movimiento-suxedsmico}{%
\subsubsection*{Carcterización del movimiento sísmico}\label{carcterizaciuxf3n-del-movimiento-suxedsmico}}
\addcontentsline{toc}{subsubsection}{Carcterización del movimiento sísmico}

En un planteo paramétrico, los modelos de predicción del movimiento sísmico se basan en estimadores no-sesgados de la mediana condicional \(\eta_{I|M,R}(m,r)\) y la varianza \(\sigma_{ln \ I|M,R}\) de una medida escalar de intensidad \(I=i^*\). Los modelos escalares requiere la elección de una medida óptima de intensidad para caracterizar la relación entre el movimiento sísmico y la demanda sísmica.

La identificación de IM óptimas puede ser abordada a partir de los conceptos de eficiencia y suficiencia. Una medida de intensidad suficiente es aquella para la cual la demanda condicional es condicionalmente independientes de la magnitud \(M\) y la distancia \(R\) de los sismos seleccionados. Para una IM suficiente, la adición de registros de diferentes M o R no reduce la variabilidad \(\sigma_{ln D|I}\). Una IM eficiente por otra parte, es aquella que reporta una variabilidad relativamente pequeña en la demanda sísmica y puede ser cuantificada en el desvío estándar del error aleatorio \(\sigma_{ln D|I}\).

Los modelos de predicción del movimiento sísmico se ha concentrado históricamente en la predicción de la aceleracion máxima del terreno \(PGA\) y en los últimas dos décadas, se ha generalizado el uso de modelos basadas en ordenadas espectrales \(Sa(T)\) de un espectro de respuesta elástico de un oscilador equivalente de un grado de libertad SDOF. En los sistemas dinámicos que pueden asemejarse a un SDOF las aceleraciones espectrales \(Sa(T)\) suelen ser medidas eficientes y suficientes para estimar la demanda sísmica en términos de fuerzas y deformaciones elásticas para un cierto rango de períodos. Las estructuras SDOF fuera del régimen elástico, o geoestructuras como taludes y presas de materiales sueltos o relaves, son sistemas dinámicos cuyas propiedades dinámicas varían durante el proceso de deformación y no se disponen medidas de intensidad (escalares) eficientes ni suficientes para estimar la respuesta sísmica

En los sistemas dinámicos en general no es posible identificar una medida de intensidad escalar que sea eficiente y suficiente y la demanda sísmica dependerá de varios aspectos del movimiento sísmico tales como la duración, el contenido de frecuencias de la señal, la energía, etc. Cuando se formulan modelos de estimación basados en un vector de medidas de intensidad \(I=\{I_1,I_2...\}\) la varianza condicional \(\sigma_{ln D|I_1,I_2,...}^2\) se reduce considerablemente. En el caso de un planteo paramétrico, la caracterización de las funciones de distribución requiere conocer la mediana y varianza condicional \(\mu_{ln \ I_1,I_2,...|M,R}(m,r)\) y \(\sigma_{ln \ I_1,I_2,...|M,R}\), y además la matriz de covarianzas \(\rho_{i,j}\) que existen entre diferentes medidas de intensidad \(\{I_i,I_j \}\) Cuando se emplean más de dos IMs, esta operatoria se vuelve muy engorrosa y por otra parte limita la aplicabilidad del modelo, ya que dos IMs eficientes para un problema dinámico en particular, difícilmente lo sean para otro.

\hypertarget{respuesta-del-sitio}{%
\subsubsection*{Respuesta del Sitio}\label{respuesta-del-sitio}}
\addcontentsline{toc}{subsubsection}{Respuesta del Sitio}

Los registros sísmicos de diferentes regiones del mundo, fueron obtenidos en estaciones sismológicas construidas sobre sitios con diferentes configuraciones geotécnicas.

Los modelos de predicción del movimiento sísmico en general incorporan factores de amplificación de ordenadas espectrales para considerar los efectos locales del sitio en la predicción de la intensidad , dependiente de categorías de suelo basadas en velocidades de onda de corte de los últimos 30 m del estrato. La caracterizacion de los efectos de sitio mediante un único factor basado en categorías de suelo, introduce una gran variabilidad epistémica en la predicción de la respuesta dinámica del sitio y la predicción del movimiento sísmico en superficie, particularmente fuerte en sitios con velocidades de onda de corte promedio menores a 500-700 m/s.

Para reducir esta variabilidad se requiere primermente caracterizar la respuesta del sitio mediante un sistema de clasificación que incorporen otros parámetros escalares mejor correlacionados con las propiedades dinámicas del suelo \(\pmb S = \{ S_1, S_2, ... \}\) Y luego, corregir los parámetros que caracterizan el movimiento sísmico en roca basal. Esto equivale en la prácita a introducir un nuevo proxy en la estimación de la demanda,

\hypertarget{ergodicidad-y-homocedasticidad}{%
\subsubsection*{Ergodicidad y Homocedasticidad}\label{ergodicidad-y-homocedasticidad}}
\addcontentsline{toc}{subsubsection}{Ergodicidad y Homocedasticidad}

El modelo de demanda sísmica \(G_D(d^*)\) asume que la variable aleatoria condicionada \(\epsilon_{I}(i^*,m,r)\) tiene una distribución normal con parámetros \(\sigma_{ln I}^2\) y \(\mu_{ln \ I|M,R}(m,r)\) constantes para un escenario sísmico \(\{M,R\}\) dado. Es decir, supone un proceso aleatorio en la cual estos parámetros no varían en el tiempo y por otra parte, que son invariantes del registro sísmico. En otras palabras, el empleo de modelos paramétricos para caracterizar las variables aleatorias \(I/M,R\) implica en la práctica que todas las series temporales (aleatorias) de la región analizada, son una realización de un único proceso aleatorio ergódico que mantiene invariantes las propiedades de sus funciones de distribución.

La ergodicidad desempeña un papel fundamental en las estimaciones de las magnitudes estadísticas, ya que garantiza que los momentos \(\sigma_{ln I}^2\) y \(\mu_{ln \ I|M,R}(m,r)\) de las series temporales sean estimadores insesgados de la esperanza y la varianza de las funciones de distribución \(G_{I|M,R}\)

Los modelos de predicción del movimiento sísmico \(\mu_{ln \ I|M,R}(m,r)\) se ajustan a través de registros de aceleraciones correspondientes a diferentes escenarios sísmicos de magnitud y distancia \(\{M,R\}\) En una situación ideal, un conjunto de registros sísmicos de aceleraciones de una región con una larga historia de sismicidad instrumental, abarcarían todos los escenarios \(\{M,R\}\) requeridos para el modelo empírico. En la práctica, dificilmente una región dispone de un número suficiente de registros sísmicos, particularmente de movimiento fuerte, que aporten estos escenarios y generalmente se emplean datasets que incorporan registros de diferentes regiones del mundo, que comparten alguna caracerística con la región de análisis. Esta limitación en la selección de los datasets incorpora una gran incertidumbre aleatoria ya que cualquier diferencia que exista entre el medio de propagación, los espectros de frecuencias de la fuente, la respuesta del sitio, etc, se manifestará como una mayor varianza \(\sigma_{ln I}^2\).

La varianza del modelo de estimación del movimiento sísmico \(\sigma_{ln \ I}\) es una constante que no depende del escenario sísmico de magnitud y distancia \(\{M,R\}\) Esto es una consecuencia de haber adotpado un modelo paramétrico que asumió una cierta forma funcional para la función de distribución (normal). Los modelos GMPE más avanzados, tienen términos de error no menores a \(\sigma_{ln \ I}\approx0.6\) lo que equivale en la práctica a asumir una variabilidad no menor al 80\% para la predicción de la intensidad en un escenario dado. Para reducir esta variabilidad se requieren relajar la hipótesis de homocedasticidad mediante una varianzas condicionales variables (GARCH) y modelos no-paramétricos, en los cuales no se requiere de antemano ninguna forma funcional explícita para las funciones de distribución (distribution-free models) del modelo \(G_{I|R,M}(i,r,m)\) y \(G_{D|I}(d^*,i)\)

\hypertarget{selecciuxf3n-de-sismos}{%
\subsubsection*{Selección de Sismos}\label{selecciuxf3n-de-sismos}}
\addcontentsline{toc}{subsubsection}{Selección de Sismos}

El estado de la práctica actual, se basa en un modelo de predicción de intensidad y demanda para sistemas elásticos con respuesta sísmica conocida. En los sistemas dinámicos más complejos, como las geoestructuras de materiales sueltos, o incluso estructuras similares a un SDOF fuera del régimen elástico, la respuesta sísmica depende en general del contenido de frecuencias y duración de los registros sísmicos, es decir dependen del registro sísmico.

En estos sistemas, la respuesta sísmica sólo puede obtenerse mediante modelos numéricos y la caracterización de la función de distribución de la demanda sísmica \(G_{D|I}(d,i^*)\) se obtiene a partir del análisis de un número grande de registros sísmicos escalados a la intensidad sísmicica objetivo \(I=i^*\).

Sin embargo, no es posible desagregar de la intensidad condicional \(I|M,R\) una muestra representativa de la demanda sísmica (un registro), y en teoría se requeriría el análisis de un gran número de regisros sísmicos para obtener los parámetros de la función de distribución de la demanda sísmica \(G_{D|I}(d,i^*)\) . En la práctica, debido principalmente al costo computacional, sólo un número reducido de registros sísmicos suele analizarse y los estimadores son sesgados y se requieren metodologías de selección de registros sísmicos para identificar aquellos que controlan la demanda.

\hypertarget{resumen}{%
\subsubsection*{Resumen}\label{resumen}}
\addcontentsline{toc}{subsubsection}{Resumen}

\begin{itemize}
\item
  La variabilidad de la demanda y en consecuencia, la incertidumbre de la probabilidad de falla de un diseño, está controlada por el error del modelos de predicción del movimiento sísmico y el error de predicción de la demanda sísmica.
\item
  La caracterización del movimiento sísmico en términos de una medida escalar de intensidad, limita la eficiencia de cualquier modelo de predicción de la demanda sísmica, particularmente en problemas que no son asemejables a un SDOF en respuesta elástica
\item
  La hipótesis de normalidad en las funciones de distribución de la intensidad (escalar) condicional, implica una varianza constante (homocedásticidad) para todos los escenarios sísmicos y afecta fuertemente el término de error aleatorio del movimiento sísmico, particularmente cuando se combinan registros sísmicos de diferentes regiones del mundo
\item
  Los efectos de sitio sólo son pueden ser considerados a partir de un factor de amplificación empírico de las ordenadas espectrals basado en la velocidad de corte de los últimos 30 m del estrato \(\tilde S_a(T) \approx AF(V_{S.30}) \ S_a(T)\) , el cual no permite correlacionar adecuadamente los efectos del sitio con el contenido de frecuencias de los registros sísmicos en superficie, y aporta una gran variabiliad adicional en las caracteristicas del movimiento sísmico en superficie, particularmente cuando existen estratos profundos de suelos de baja rigidez sometidos a movimiento fuerte del terreno.
\item
  En los sistemas dinámicos en general, los estimadores de la demanda sísmica condicional \(D|I\) son siempre sesgados,y el sesgo dependerá fuertemente del número de registros sísmicos que puedan analizarse.
\item
  Cuando los sistemas dinámicos son estructuras complejas, la cantidad de registros sísmicos a analizar está fuertemente limitada por el costo computacional del modelo numérico y se requieren técnicas especializas para seleccionar registros sísmicos para obtener estimadores sesgados (con menos sesgo) de la mediana condicional \(\eta_{D|I}(i)\)
\end{itemize}

En los parágrafos siguientes, se presenta un enfoque a nivel conceptual, que pretende resolver los desafíos que presentan los puntos anteriores

\hypertarget{enfoque-propuesto}{%
\subsection*{Enfoque propuesto}\label{enfoque-propuesto}}
\addcontentsline{toc}{subsection}{Enfoque propuesto}

El propósito del presente trabajo de investigación es la formulación de una metodología general de estimación de respuesta sísmica de sistemas dinámicos complejos, basada en sismos característicos de diseño de la región y el sitio.

Para llevar a cabo este propósito se requiere en primer lugar, caracterizar el movimiento sísmico en la roca basal (es decir, desagregando la influencia del sitio) mediante series temporales parametricas, de modo tal de obtener una función de distribución en función de un conjunto reducido de parámetros \(\pmb \psi = \{ \psi_1,\psi_2,... \}\), que serán característicos de la región. Según este enfoque, los parámetros dinámicos que caracterizan los sismos de una región dada, podrán obtenerse según la siguiente función de distribución. \[ \begin{equation}  
f_{\Psi}(\psi_1^*,\psi_2^*,...) \approx \int^{r_{max}^*}_{r_{min}^*}  \int^{m_{max}^*}_{m_{min}^*} f_{\Psi|R,M}(\psi_1^*,\psi_2^*,...,r,m) \  f_{R|M}(r,m)\  f_M(m) \ dm \ dr 
\end{equation}\]

Suponemos que dentro del universo de registros sísmicos en roca basal \(\pmb\Psi = \{ \Psi_1,\Psi_2,... \}\), existe un subconjunto de registros sísmicos asociados a la región \(\pmb\Psi^L = \{ \Psi_1^L,\Psi_2^L,... \}\) que tienen la propiedad de ser mutuamente independientes \emph{i.i.d.} Luego, la caracterización de una región sísmica será reducirá a poder obtener numéricamente la función de distribución de cada parámetro local \(\Psi_k^{L}\)

\[ \begin{equation}  
f_{\Psi}\left(\pmb \psi^L\right) \approx \int^{r_{max}^*}_{r_{min}^*}  \int^{m_{max}^*}_{m_{min}^*} \prod^{n_p}_{k=1}{f_{\Psi_k|R,M}(\psi_k^L,r,m)} \  f_{R|M}(r,m)\  f_M(m) \ dm \ dr \approx  \prod^{n_p}_{k=1}{f_{\Psi_k}(\psi_k^L)}
\end{equation}\]

El paso siguiente es incluir los efectos del terreno en los parámetros \(\pmb\Psi^{S,L}=\{\psi_1^{S,L},\psi_2^{S,L},... \}\) a partir de un modelo empírico de respuesta dinámica del sitio, que incluya como variables independientes algunas propiedades dinámicas características como el período natural del sitio \(T_S\), el espesor del estrato \(H_S\) y la rigidez al corte promedio \(G_S\).

La influencia de los efectos de sitio, puede ser considerada o bien en los modelos de predicción del movimiento sísmico local, o bien en los modelos de estimación de la demanda sísmica. En el primer caso, se requiere condicionar el movimiento sísmico al sitio mediante una variable aleatoria \(\pmb\Theta|\pmb\Psi,\pmb S\) y la función de distribución del movimiento sísmico en terreno podría formularse a partir de una mediana condicional \(\eta_{\Theta|\Psi,S}\) según

\[ \begin{equation}
\Theta|\Psi,S \approx \eta_{\Theta|L,S}(\psi_1^{L},\psi_2^{L},...,T_S,V_S,H_S) \ \epsilon_{\Theta}
\end{equation}\]

donde \(\epsilon_{\Theta}\) es una variable aleatoria normal con media unitaria y desvío estándar \(\sigma_{\Theta|\Psi,S}\) Según esta estrategia, las propiedades geotéctnicas de cada sitio \(\pmb {s^*}=\{T_S,V_S,H_S\}\) determinan una nueva función de distribución \(f_{\Theta|{\Psi,S}}(\theta,\psi,s^*)\) para los movimientos sísmicos en superficie de una región \(L\) y un sitio \(S\) y la demanda sísmica estará condicionada por los sismos en superficie \(D|\Theta\) según una función de distribución

\[ \begin{equation} 
G_D(d^*,\pmb {s^*}) \approx \sum^{}_{all \ \pmb\theta^L} \ \sum^{}_{all \ \pmb\psi^L} \  G_{D|\Theta} \left( d^*,\pmb\theta \right ) \ f_{\Theta|\Psi,S}\left (\pmb\theta,\pmb\psi,\pmb {s^*}  \right ) \  \prod^{n_p}_{k=1}{f_{\Psi_k}(\psi_k^L)} 
\end{equation} \]

Esta alternativa, requiere un factor que permita clasificar los efectos del sitio en una única variable aleatoria

La segunda alternativa para considerar los efectos del sitio es condicionar directamente la demanda según una variable aleatoria condicional \(D|S\). En los sistemas dinámicos complejos, la demanda sísmica sólo puede estimarse mediante modelos numéricos FEM cuyos resultados pueden parametrizarse a partir de medianas condicionales, y esta alternativa es la opción más eficiente, ya que elimina al proxy de respuesta de sitio. Cuando además se conocen las propiedades geotécnicas de todos los estratos desde el nivel de fundación hasta roca basal, siempre es posible incorporar al suelo como parte del sistema dinámico. Asumiendo que disponemos de un modelo numérico de estas caraterísticas, la demanda sísmica condicional queda determinada según una variable aleatoria \(D|\pmb\Psi,\pmb S\) que tendrá una función de distribución \(G_{D|\Psi,S}(d,\pmb\psi,\pmb s^*)\) y la demanda sísmica probabilística quedará determinada según

\[ \begin{equation} 
G_D(d^*,\pmb {s^*}) \approx \sum^{}_{all \ \pmb\psi^L} \  G_{D|\Psi,S} \left( d^*,\pmb\psi^L, \pmb s^* \right ) \  \prod^{n_p}_{k=1}{f_{\Psi_k}(\psi_k^L)} 
\end{equation} \]

donde \(\psi^L = \{ \psi_1^L,\psi_2^L,... \}\) son los sismos en roca basal característicos de la región. En los parágrafos que siguen, se identifican los objetivos principales que se requieren para llevar a cabo la estrategia numérica planteada.

\hypertarget{objetivos}{%
\subsection*{Objetivos}\label{objetivos}}
\addcontentsline{toc}{subsection}{Objetivos}

El propósito del presente trabajo es desarrollar una metodología robusta para la estimación de la demanda sísmica probabilística de sistemas dinámicos complejos y tendrá dos objetivos fundamentales:

\hypertarget{caracterizaciuxf3n-del-movimiento-suxedsmico}{%
\subsubsection*{Caracterización del Movimiento Sísmico}\label{caracterizaciuxf3n-del-movimiento-suxedsmico}}
\addcontentsline{toc}{subsubsection}{Caracterización del Movimiento Sísmico}

El primer objetivo se enfocará en los datos y consistirá en establecer metodologías de caracterización y clasificación de registros sísmicos, basadas en modelos dinámicos de series paramétricas. Las metodologías de caracterización deberá basarse en el análisis de millones de acelerogramas en roca basal, que puedan ser desagregados segun diferentes escenarios sísmicos de intensidad, y requieren la formulación de bases de datos de movimiento sísmico en roca basal. Las metodologías de clasificación deberán permitir identificar registros sísmicos de un sitio particular y poder asociarlo a un grupo (cluster) de sismos característicos de regiones similares.

\hypertarget{predicciuxf3n-de-la-demanda-suxedsmica}{%
\subsubsection*{Predicción de la Demanda Sísmica}\label{predicciuxf3n-de-la-demanda-suxedsmica}}
\addcontentsline{toc}{subsubsection}{Predicción de la Demanda Sísmica}

El segundo objetivo se enfocará en la predicción y consistirá en establecer los requisitos que deberá un modelo robusto de predicción del movimiento y la demanda sísmica que incluya los sismos de diseño de una región y las características geotécnicas particulares de un sitio dado. En esta etapa se requieren metodologías específicas que permitan integrar la respuesta dinámica de modelos numéricos complejos FEM para la estimación de la respuesta sísmica de sistemas complejos

\hypertarget{organizaciuxf3n}{%
\subsection*{Organización}\label{organizaciuxf3n}}
\addcontentsline{toc}{subsection}{Organización}

El trabajo de investigación estará organizado en cuatro partes.

La primera parte (A) presenta los capítulos preliminares como la Introducción y el Estado del Arte (CApítulo I), los conceptos preliminares y antecedentes (Capítulo II) y finalmente un problema de aplicación simple, que pone en evidencia las principales limitaciones e implicancias del estado actual del arte en la estimación de la demanda sísmica

La segunda parte (B) estará enfocada básicamente en el tratamiento y la clasificación de las series temporales y se desarrolla en los capítulos I a V.

La tecera parte del trabajo estará enfocada en la predicción del movimiento sísmico y la demanda sísmica y abarca los capítulos VI a VIII

La cuarta y última parte presenta primeramente (Capítulo IX) un problema de aplicación de la práctica donde se resumen y discuten los resultados obtenidos, y luego, en el capítulo X, se presentan los aportes y limitaciones de las metodologías propuestas y se proponen las líneas futuras de investigación basadas en este modelo.

\hypertarget{parte-b-caracterizaciuxf3n-del-movimiento-suxedsmico}{%
\subsubsection*{Parte B: Caracterización del Movimiento Sísmico}\label{parte-b-caracterizaciuxf3n-del-movimiento-suxedsmico}}
\addcontentsline{toc}{subsubsection}{Parte B: Caracterización del Movimiento Sísmico}

El capítulo I se ocupa del minado de datos y el control de calidad de los datos. En este capítlo el objetivo es compilar catálogos y bases de datos de diferentes regiones del mundo. En esta etapa se identificarán los registros válidos a partir de un anállisis de integridad y calidad de los metadatos y se eliminarán todos los registros que no cumplan con ciertos criterios mínimos de aceptación (QC) basados básicamente en la cantidad de metadatos asociados al registro, tales como las condiciones geotécnicas de la estación de registro, los datos de magnitud y distancias del evento sísmico y . El aporte de este capítulo serán una base de datos relacional eventos, estaciones, y registros sísmicos asociados a cada evento y estación.

El capítulo II, plantea los algoritmos fundamentales requeridos para el procesamiento de series temporales asociadas a eventos śismicos. El procesamiento incluirá el resampleo, la aplicación de filtros pasa-banda, la eliminación de valores medios (DC) , tendencias (trends) y saltos (glitch), el recorte de las duraciones efectivas y la implementación de algoritmos robustos de integración de las series temporales. El aporte de este capítulo será una tabla de registros sísmicos normalizados en superficie y un algoritmo (librería) de código abierto para el procesamiento de registros sísmicos

El capítulo III se efocará en remover la respuesta dinámica del sitio de los registros sísmicos en superficie y consistirá básicamente en generar registros sísmicos en roca basal a partir de la simulación numérica de todos los posibles perfiles geotécnicos que puedan existir bajo una estación sismológica en particular, en la cual se conocen un cierto número de parámetros geotécnicos y dinámicos. A partir de los diferentes perfiles geotécnicos simulados, se impementará un sistema de clasificación de sitios a partir de un factor que combine la respuesta dinámica del sitio y las propiedades dinámicas del perfil geotécnico

El capítulo IV es el capítulo clave en donde se buscará la parametrización de las series temporales asociadas a cada evento sísmico. El objetivo principal de este capítulo será la identificación de los parámetros asociados a un modelo dinámico equivalente \(\psi = \{ \psi_1,\psi_2,... \}\) mediante el análisis paramétrico de series temporales no-estacionarias en roca basal. Esta es la etapa define uno de los aportes más importantes del trabajo, ya que establece una representación paramétrica de los registros sísmicos regularizados y normalizados en roca basal

\hypertarget{capuxedtulo-iv-clustering-corregir}{%
\subsubsection*{Capítulo IV: Clustering ** corregir}\label{capuxedtulo-iv-clustering-corregir}}
\addcontentsline{toc}{subsubsection}{Capítulo IV: Clustering ** corregir}

El objetivo del capítulo II consistirá en desagregar los parámetros dinámicos del modelo equivalente \(\psi = \{ \psi_1,\psi_2,... \}\) en diferentes escenarios sísmicos de magnitud y distancia a la fuente \(\{M,R\}\) y estimar un conjunto de medidas de intensidad asociadas a cada escenario sísmico. El objetivo de este capítulo será formular un modelo de predicción del movimiento sísmico sencillo en roca basal basado en intensidades

La clasificación de los datos, que analizará los diferentes conjuntos de parámetros dinámicos \(\psi = \{ \psi_1,\psi_2,... \}\) e identificará grupos (clústers) de registros sismicos (desagregados por escenarios) mediante la implementación de un conjunto de métricas adecuadas para series temporales paramétricas. Este objetivo será abordado en el capítulo 4

\hypertarget{parte-ii-predicciuxf3n-de-la-demanda-suxedsmica}{%
\subsubsection*{Parte II: Predicción de la Demanda Sísmica}\label{parte-ii-predicciuxf3n-de-la-demanda-suxedsmica}}
\addcontentsline{toc}{subsubsection}{Parte II: Predicción de la Demanda Sísmica}

Formular un modelo simple de predicción del movimiento sísmico a nivel de roca basal en términos de los parámetros . Para cada escenario sísmico, el objetivo de esta etapa será obtener las funciones de distribución del movimiento sísmico \(f_{\Psi^L|M,R}\left(m,r \right)\) expresado en términos de los sismos característicos de la región \(\Psi^L = \{ \Psi_1^L,\Psi_2^L,... \}\). Este objetivo será abordado en el capítulo 5

\begin{itemize}
\tightlist
\item
  Definir un modelo general de predicción de la demanda sísmica basada en los sismos característicos de la regi \(\pmb\Psi^L\)y en un factor de sitio \(S\). Este objetivo será abordado en el capítulo 7
\end{itemize}

\hypertarget{section}{%
\subsubsection*{}\label{section}}
\addcontentsline{toc}{subsubsection}{}

\hypertarget{aportes}{%
\subsection*{Aportes}\label{aportes}}
\addcontentsline{toc}{subsection}{Aportes}

El principal aporte de este trabajo será una metodología de generación automática de modelos de predicción de respuesta sísmica, basados en aprendizaje automático no-supervisdo.

El aporte principal del trabajo de investigación será la identificación de un conjunto reducido de parámetros dinámicos \emph{i.i.d.} que caracterizan los movimientos sísmicos de una región y determina toda una familia de métodos de predicción. Si bien la caracterización paramétrica de series temporales es un concepto que data de la década del '80, la disponibilidad de una base de datos de cientos de miles de registros sísmicos de diferentes regiones del mundo, permitirá emplear metodologías basadas en aprendizaje estadístico (machine-learning) no supervisado que aporten nueva luz sobre un problema extremadamente complejo de abordar a través de modelos físicos.

Desde el punto de vista de los modelos, el empleo de métodos de predicción no-paramétricos (distribution-free) y la desagregación de los efectos del sitio de la mediana condicional y la varianza, aporta un modelo simple de predicción del movimiento sísmico en términos de las series temporales esperables en un sitio y región dados

Desde el punto de vista de los datos, el procesamiento y análisis de series temporales, aportará una base de datos de acceso público de varios millones de registros sísmicos reales e híbridos (simulados), que incluirán como característica novedosa los metadatos asociados a los parámetros dinámicos (features) de cada registro y los sistemas de clasificación de cada grupo (Región, Mecanismo, etc. ). Por otra parte, la publicación de datos en formatos de librerías y datasets de código abierto en lenguaje R, facilitará el desarrollo de nuevo modelos de predicción más sofisticados que incorporen nuevos datos sísmicos en futuras líneas de investigación basadas en esta familia de modelos

Alejandro Verri Kozlowski. Mendoza, Abril de 2021

\end{document}
